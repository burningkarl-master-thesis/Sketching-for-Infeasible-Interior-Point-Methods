% Adapted from https://jeffe.cs.illinois.edu/pubs/latex.html: jeffe.sty
%   originally designed by Jeff Erickson (jeffe@cs.uiuc.edu)
% Adapted from https://github.com/vinh-tr-hoang/MathNotations
%   originally designed by Truong Vinh Hoang
% Adapted by Karl Welzel on 24 February 2021

\usepackage{mathtools}
\usepackage{amsthm}
\usepackage{amssymb}
\usepackage{stmaryrd}
\SetSymbolFont{stmry}{bold}{U}{stmry}{m}{n}  % See https://tex.stackexchange.com/questions/106714/stmaryrd-and-boldsymbol-avoid-warnings
\usepackage{bm}  % provides \bm for bold math. Better than \boldsymbol
\usepackage{mleftright} \mleftright{}  % improves spacing for \left and \right
\usepackage{braket}  % provides \Set with scaled middle |
\usepackage{siunitx}  % units (SI)

\usepackage[
 ruled,
 vlined,
 linesnumbered,
]{algorithm2e}

% ----------------------------------------------------------------------
%  Common abbreviations and words with accents
% ----------------------------------------------------------------------

\hyphenation{co-or-din-ate co-or-din-ates half-plane half-space stereo-iso-mers
stereo-iso-mer Round-table homol-ogous homol-ogy}

% ---- LATIN ----
\newcommand{\etal}{\emph{et~al.}}			% and others
\newcommand{\ie}{i.e.}						% that is
\newcommand{\eg}{e.g.}						% for example
\newcommand{\vitae}{vit\ae{}}
\newcommand{\apriori}{\emph{a~priori}}
\newcommand{\aposteriori}{\emph{a~posteriori}}

% ---- FRENCH ----
\newcommand{\naive}{na\"{\i}ve}
\newcommand{\Naive}{Na\"{\i}ve}
\newcommand{\naively}{na\"{\i}vely}	% Okay, I know, this isn't French.
\newcommand{\Naively}{Na\"{\i}vely}
\newcommand{\cafe}{caf\'e}

% ---- GERMAN ----
\newcommand{\fur}{f\"ur}
\newcommand{\Universitat}{Universit\"at}
\newcommand{\Saarbrucken}{Saar\-br\"ucken}	% Bypass TeX hyphenation
\newcommand{\Zurich}{Z\"urich}

% ---- PORTUGESE (Hi Jorge!) ----
\newcommand{\Computacao}{Computa\c{c}\~ao}

% ---- PROPER NAMES (because I'm lazy) ----
\newcommand{\Benes}{Bene\v{s}}		% ...network
\newcommand{\Bezier}{B\'ezier}		% ...spline/curve/surface
\newcommand{\Bjorner}{Bj\"orner}
\newcommand{\Bochis}{Bochi\c{s}}		% Daciana
\newcommand{\Boruvka}{Bor\.uvka}		% ...'s MST algorithm
\newcommand{\Bragger}{Br\"agger}
\newcommand{\Bronnimann}{Br\"onnimann}   \newcommand{\Herve}{Herv\'e}
\newcommand{\Bruckner}{Br\"uckner}
\newcommand{\Caratheodory}{Carath\'eodory}	% Constantin
\newcommand{\Chvatal}{Chv\'atal}         \newcommand{\Vasek}{Va\v{s}ek}
				\newcommand{\Joao}{Jo\~ao}	% Compa
\newcommand{\Cortes}{Cort\'es}		% Carmen
\newcommand{\Dujmovic}{Dujmovi\'c}	% Vida
				\newcommand{\Fredo}{Fr\'edo}	% Durand
\newcommand{\Erdos}{Erd\H{o}s}           \newcommand{\Pal}{P\'al}
\newcommand{\Frechet}{Fr\'echet}
\newcommand{\Furedi}{F\"uredi}           \newcommand{\Zoltan}{Zolt\'an}
\newcommand{\Grobner}{Gr\"obner}		% ... basis
\newcommand{\Grunbaum}{Gr\"unbaum}	% Branko
\newcommand{\Hanoi}{Hano\"\i}		% Tower of...
\newcommand{\Jarnik}{Jarn\a'{\i}k}	% ...'s (`Prim's') MST algorithm
\newcommand{\Komlos}{Koml\'os}
\newcommand{\Kovari}{K\"ov\'ari}
\newcommand{\Lovasz}{Lov\'asz}           \newcommand{\Laszlo}{L\'aszl\'o}
\newcommand{\Matousek}{Matou\v{s}ek}     \newcommand{\Jiri}{Ji\v{r}\'\i}
\newcommand{\Mnev}{Mn\"ev}
\newcommand{\Mobius}{M\"obius}		% ... strip/transformation/function
\newcommand{\Mucke}{M\"ucke}		% Ernst
\newcommand{\ODunliang}{\'O'D\'unliang}
\newcommand{\Oleinik}{Ole\u{\i}nik}
                                \newcommand{\Janos}{J\'anos}     % Pach
\newcommand{\Palasti}{Pal\'asti}
				\newcommand{\Belen}{Bel\'en}	% Palop
\newcommand{\Petrovskii}{Petrovski\u{\i}}
\newcommand{\Pinar}{P\i nar}		% Ali
\newcommand{\Plucker}{Pl\"ucker}		% ... coordinates
\newcommand{\Poincare}{Poincar\'e}	% ... duality/halfplane
                                \newcommand{\Gunter}{G\"unter}   % Rote, Ziegler
\newcommand{\Sacristan}{Sacrist\'an}	% Vera
\newcommand{\Saskin}{\v{S}a\v{s}kin}
\newcommand{\Schomer}{Sch\"omer}
\newcommand{\Schonhardt}{Sch\"onhardt}	% ... polyhedron
\newcommand{\Sos}{S\'os}
\newcommand{\Stackel}{St\"ackel}		% Paul
\newcommand{\Szekely}{Sz\'ekely}
\newcommand{\Szemeredi}{Szemer\'edi}
\newcommand{\Toth}{T\'{o}th}		% Geza
\newcommand{\Turan}{Tur\'an}
\newcommand{\Ungor}{\"Ung\"or}		% Alper
\newcommand{\Voronoi}{Vorono\"\i}		% ... diagram [for francophile pedants only]

% Other
\newcommand{\Cplusplus}{C\raisebox{0.5ex}{\tiny\textbf{++}}}

% ----------------------------------------------------------------------
%  Simple math stuff
% ----------------------------------------------------------------------

% ---- SYMBOLS ----
\let\e\varepsilon{}               % a ``real'' epsilon
\newcommand{\Probability}{\mathbb{P}}

\newcommand{\Real}{\mathbb{R}}
\newcommand{\Proj}{\mathbb{P}}
\newcommand{\Hyper}{\mathbb{H}}
\newcommand{\Integer}{\mathbb{Z}}
\newcommand{\Natural}{\mathbb{N}}
\newcommand{\Complex}{\mathbb{C}}
\newcommand{\Rational}{\mathbb{Q}}

\let\N\Natural{}
\let\Q\Rational{}
\let\R\Real{}
\let\Z\Integer{}
\newcommand{\Rd}{\Real^d}
\newcommand{\RP}{\Real\Proj}
\newcommand{\CP}{\Complex\Proj}

\newcommand*\diff{\mathop{}\!\mathrm{d}}	% Inside of integrals

% ---- OPERATORS ----
\DeclareMathOperator{\aff}{aff}
\DeclareMathOperator{\area}{area}
\DeclareMathOperator{\argmax}{arg\,max}
\DeclareMathOperator{\argmin}{arg\,min}
\DeclareMathOperator{\Aut}{Aut}			% Automorphism group
\DeclareMathOperator{\card}{card}		% cardinality, deprecated for \abs
\DeclareMathOperator{\conv}{conv}
\DeclareMathOperator{\Hom}{Hom}			% Homomorphism group
\DeclareMathOperator{\id}{id}			% identity
\DeclareMathOperator{\im}{im}			% image
\DeclareMathOperator{\lcm}{lcm}
\DeclareMathOperator{\poly}{poly}
\DeclareMathOperator{\polylog}{polylog}
\DeclareMathOperator{\rank}{rank}
\DeclareMathOperator{\rel}{rel\,}		% relative (interior, boundary, etc.)
\DeclareMathOperator{\sgn}{sgn}
\DeclareMathOperator{\vol}{vol}			% volume
\DeclareMathOperator{\nnz}{nnz}			% # nonzero entries in a matrix
\DeclareMathOperator{\diag}{diag}       % diagonal matrix

\newcommand{\fp}[1]{^{\underline{#1}}}	% falling powers: $n\fp{d}$
\newcommand{\rp}[1]{^{\overline{#1}}}	% rising powers:  $n\rp{d}$


% --- Darts and fences ---
\newcommand{\arcto}{\mathord\shortrightarrow}
\newcommand{\arc}[2]{#1\arcto#2}
\newcommand{\cra}[2]{#1\mathord\shortleftarrow#2}
\newcommand{\fence}[2]{#1\mathord\shortuparrow#2}
\newcommand{\ecnef}[2]{#1\mathord\shortdownarrow#2}

% --- Cheap displaystyle operators ---
\newcommand{\Frac}[2]{{\displaystyle\frac{#1}{#2}}}
\newcommand{\Sum}{\sum\limits}
\newcommand{\Prod}{\prod\limits}
\newcommand{\Union}{\bigcup\limits}
\newcommand{\Inter}{\bigcap\limits}
\newcommand{\Lor}{\bigvee\limits}
\newcommand{\Land}{\bigwedge\limits}
\newcommand{\Lim}{\lim\limits}
\newcommand{\Max}{\max\limits}
\newcommand{\Min}{\min\limits}

% ---- RELATORS ----
\newcommand{\deq}{\coloneqq}	% Use := instead.
\newcommand{\into}{\DOTSB\hookrightarrow}		% = one-to-one
\newcommand{\onto}{\DOTSB\twoheadrightarrow}
\newcommand{\inonto}{\DOTSB\lhook\joinrel\twoheadrightarrow}
\newcommand{\from}{\leftarrow}
\newcommand{\tofrom}{\leftrightarrow}

% ---- DELIMITER PAIRS ----
\newcommand{\floor}[1]{\lfloor#1\rfloor}
\newcommand{\ceil}[1]{\lceil#1\rceil}
\newcommand{\seq}[1]{\langle#1\rangle}
\newcommand{\abs}[1]{\lvert#1\rvert}		% use instead of $|x|$ 
\newcommand{\norm}[1]{\lVert#1\rVert}		% use instead of $\|x\|$ 
\newcommand{\indic}[1]{\big[#1\big]}		% indicator variable; Iverson notation
								% e.g., Kronecker delta = [x=0]

% --- Self-scaling delmiter pairs ---
\newcommand{\Floor}[1]{\left\lfloor#1\right\rfloor}
\newcommand{\Ceil}[1]{\left\lceil#1\right\rceil}
\newcommand{\Seq}[1]{\left\langle#1\right\rangle}
\newcommand{\Abs}[1]{\left\lvert#1\right\rvert}
\newcommand{\Norm}[1]{\left\lVert#1\right\rVert}
\newcommand{\Paren}[1]{\left(#1\right)}		% need better macro name!
\newcommand{\Brack}[1]{\left[#1\right]}		% need better macro name!
\newcommand{\Indic}[1]{\left[#1\right]}		% indicator variable; Iverson notation

% ----------------------------------------------------------------------
%  Maths environments (using amsthm)
% ----------------------------------------------------------------------

% The \AtBeginDocument hack is included to define those labels after including cleverref
\AtBeginDocument{
  \newtheorem{theorem}{Theorem}[chapter] % label starts with 'thm:'
  \newtheorem{lemma}[theorem]{Lemma} % label starts with 'thm:'
  \newtheorem{corollary}[theorem]{Corollary} % label starts with 'thm:'

  \theoremstyle{definition}
  \newtheorem{definition}[theorem]{Definition} % label starts with 'def:'

  \theoremstyle{remark}
  \newtheorem*{remark}{Remark} % label starts with 'rem:'
}

% ----------------------------------------------------------------------
%  Standard notation for vectors and matrices
% ----------------------------------------------------------------------

%scalar: serifs, non-bold, lowercase
%vector: serifs, bold, lowercase
\newcommand{\vek}[1]{\bm{#1}}
%matrix: serifs, bold, uppercase
\newcommand{\mat}[1]{\bm{#1}}
%tensor: sans-serifs, bold, uppercase
\newcommand{\tns}[1]{\textsf{\textbf{\textsl{#1}}}}
%random variable: serifs, non-bold, uppercase
\newcommand{\rv}[1]{#1}

