\chapter{Experiments}\label{chap:experiments}

Some practical remarks:
\begin{enumerate}
  \item \Cref{line:compute-v} of \cref{alg:newton-direction} can be simplified to directly compute \(\mat{S}^{-1}\vek{v}\) which is needed in \cref{line:compute-delta-x}.
  \item We can use a sparse QR solver in \cref{alg:newton-direction} (SPQR by Tim Davis)
  \item How to choose \(\alpha\)?
  \begin{itemize}
    \item The function \({(\vek{x}^k + \alpha \Delta\vek{x}^k)}^T (\vek{s}^k + \Delta\vek{s}^k)\) is a quadratic function in \(\alpha\) so it is easy to minimise.
    \item We do not need to minimise \(\mu(\alpha)\). Just choose \(\alpha = 0.9\).
  \end{itemize}
  \item We do not need to care about \(\vek{v}\) in practice. (Experiment with this: How do \(\norm{\vek{v}}\) and \(\norm{\vek{f}}\) behave?)
  \item To avoid nearly rank-deficient matrices we need to construct our synthetic data in a clever way
  \item Real data can be obtained from netlib or the datasets used by Avron
  \item How to terminate CG\@? Fixed number of iterations, fixed accuracy?
  \item We do not need to scrap our results if the subspace embedding fails
  \item How to choose \(\sigma\)? \(\sigma = 0.1\) should be good, cite Zhen
  \item How to choose \(w, s\)? \(m < w < 2m\) and \(s = 2, 3\) works, cite Zhen
  \item It does not matter if the implementation is indeed the long-step infeasible IPM above, an existing Mehrotra Predictor-Corrector code is also okay.
\end{enumerate}

\begin{figure}[h]
  \centering
  \import{plots/}{sketching_parameters_1.pgf}
\end{figure}

\begin{figure}[h]
  \centering
  \import{plots/}{sketching_parameters_2.pgf}
\end{figure}