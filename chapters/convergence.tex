\chapter{Convergence Proofs}\label{chap:convergence}

After the previous two chapters introduced the important ideas and algorithms, we need to establish their convergence properties.
Note that \cref{alg:ipm} is the same infeasible inexact long-step IPM as in~\cite{Monteiro-ConvergenceAnalysisLongStepInfeasibleIPMs} but with a different preconditioner.
Thus, the following proofs are similar to those in their work adapted to the new preconditioner using the ideas of \textcite{Avron-FasterRandomizedInfeasibleIPMs}.
The proofs in~\cite{Avron-FasterRandomizedInfeasibleIPMs} were often schematic, especially the runtime considerations.
The following section therefore tries to clearly lay out how the different lemmas imply \cref{thm:ipm-convergence,thm:approximate-newton-convergence} and how the runtimes for \cref{alg:ipm,alg:newton-direction} can be derived.
The essential assumptions in the following proofs are that the algorithms work with exact arithmetic, that solutions of \labelcref{eqn:primal-lp,eqn:dual-lp} exist and that \(\mat{A}\) has full row rank.

\section{Convergence Proof for the Infeasible IPM Algorithm}

Let \((\vek{x}, \vek{y}, \vek{s}) \in \mathcal{N}_{-\infty}(\gamma)\) denote a possible iterate during the course of \cref{alg:ipm}, \((\hat{\Delta\vek{x}}, \hat{\Delta\vek{y}}, \hat{\Delta\vek{s}})\) the Newton direction determined in \cref{line:compute-approximate-newton} and 
\begin{align}
  (\vek{x}(\alpha), \vek{y}(\alpha), \vek{s}(\alpha)) &\coloneqq (\vek{x}, \vek{y}, \vek{s}) + \alpha (\hat{\Delta\vek{x}}, \hat{\Delta\vek{y}}, \hat{\Delta\vek{s}}) \\
  \mu(\alpha) &\coloneqq {\vek{x}(\alpha)}^T \vek{s}(\alpha) \\
  \vek{r}(\alpha) &\coloneqq (\mat{A}\vek{x}(\alpha) - \vek{b}, \mat{A}^T \vek{y}(\alpha) + \vek{s}(\alpha) - \vek{c}).
\end{align}
Using this notation~\textcite{Monteiro-ConvergenceAnalysisLongStepInfeasibleIPMs} showed that the stepsize \(\bar{\alpha}\) determined in \cref{alg:ipm} is bounded from below.

\begin{lemma}[Lemma 3.6 in~\cite{Monteiro-ConvergenceAnalysisLongStepInfeasibleIPMs}]\label{thm:alpha-bar-bound}
  Assume that 
  \begin{itemize}
    \item \(\gamma \in (0, 1)\), \(\sigma \in (0, \frac{4}{5})\),
    \item \((\vek{x}, \vek{y}, \vek{s}) \in \mathcal{N}_{-\infty}(\gamma)\) and
    \item \((\hat{\Delta\vek{x}}, \hat{\Delta\vek{y}}, \hat{\Delta\vek{s}})\) satifies \cref{eqn:perturbed-search-direction} such that \(\norm{\vek{v}}_\infty \leq \gamma \sigma \mu / 4\).
  \end{itemize}
  Then the stepsize \(\bar{\alpha}\) determined in \cref{line:alpha-tilde,line:alpha-bar} satifies
  \[ \bar{\alpha} \geq \min \Set{1, \frac{\min \Set{\gamma \sigma, 1 - \frac{5}{4}\sigma} \mu}{4 \norm{\hat{\Delta\vek{x}} \circ \hat{\Delta\vek{s}}}_\infty}} \]
  and
  \[ \mu(\bar{\alpha}) \leq \Paren{1 - \Paren{1 - \frac{5}{4}\sigma} \frac{\bar{\alpha}}{2}} \mu. \]
\end{lemma}

Note that the assumptions in this lemma are not stated in this form in the original lemma but are instead inferred from the definition of their algorithm.
In the proof they show that \(\norm{\vek{v}}_\infty \leq \gamma \sigma \mu / 4\) and then deduce the results from it.
It suffices now to upper-bound \(\norm{\hat{\Delta\vek{x}} \circ \hat{\Delta\vek{s}}}_\infty\) to show that \(\mu\) decreases enough in each iteration.
To this end, we need a key observation which motivated us to shift the error term in \cref{chap:ipm} using the perturbation vector \(\vek{v}\):
The form of the error term in \cref{eqn:approximate-newton} ensures that at each iteration the residuals \(\vek{r} = (\vek{r}_p, \vek{r}_d)\) lie on the line segment between \(\vek{r}^0\) and \(\vek{0}\)
because \(\vek{r}(\alpha) = (1 - \alpha) \vek{r}(0)\).
In other words, \(\vek{r} = \eta \vek{r}^0\) with \(\eta \in [0, 1]\) for every iterate in \cref{alg:ipm}.

\begin{lemma}[Lemma 16 in~\cite{Avron-FasterRandomizedInfeasibleIPMs}]\label{thm:delta-x-s-bound}
  Assume that
  \begin{itemize}
    \item \(\gamma \in (0, 1)\), \(\sigma \in (0, 1)\), 
    \item \((\vek{x}^0, \vek{y}^0, \vek{s}^0) \in \mathcal{G}\) satisfies \((\vek{x}^0, \vek{s}^0) \geq (\vek{x}^*, \vek{s}^*)\) for some \((\vek{x}^*, \vek{y}^*, \vek{s}^*) \in \mathcal{F}^*\), 
    \item \((\vek{x}, \vek{y}, \vek{s}) \in \mathcal{N}_{-\infty}(\gamma)\) with \(\vek{r} = \eta \vek{r}^0\) for some \(\eta \in [0, 1]\) and
    \item \((\hat{\Delta\vek{x}}, \hat{\Delta\vek{y}}, \hat{\Delta\vek{s}})\) satifies \cref{eqn:perturbed-search-direction} such that \(\norm{\vek{v}}_2 \leq \gamma \sigma \mu / 4\).
  \end{itemize}
  Then
  \[ \norm{\mat{D}^{-1}\hat{\Delta\vek{x}}}_2, \norm{\mat{D}\hat{\Delta\vek{s}}}_2 \leq \Paren{1 + \frac{\sigma^2}{1- \gamma} - 2\sigma}^{1/2} \sqrt{n \mu} + \frac{6}{\sqrt{1 - \gamma}} n\sqrt{\mu} + \frac{\gamma \sigma}{4 \sqrt{1 - \gamma}}\sqrt{\mu}\]
  which implies that \(\norm{\mat{D}^{-1}\hat{\Delta\vek{x}}}_2\) and \(\norm{\mat{D}\hat{\Delta\vek{s}}}_2\) are bounded by \(C n \sqrt{\mu}\) for some constant \(C > 0\) depending on \(\gamma\) and \(\sigma\).
\end{lemma}

These two lemmas are enough to show the convergence of \cref{alg:ipm}.

\begin{proof}[Proof of \cref{thm:ipm-convergence}]
  By the choice of the initial point \((\vek{x}^0, \vek{y}^0, \vek{s}^0) \in \mathcal{N}_{-\infty}(\gamma)\) and \(\bar{\alpha}\) in \cref{line:alpha-tilde,line:alpha-bar} every iterate \((\vek{x}^k, \vek{y}^k, \vek{s}^k)\) is inside the specified neighbourhood \(\mathcal{N}_{-\infty}(\gamma)\).
  Therefore, \(\mu^k \leq \e_\mu \mu^0\) implies \(\norm{\vek{r}^k} \leq \e_\mu \norm{\vek{r}^0}\) and it suffices to show the former for some \(k = O(n^2 \log(1/\e_\mu))\).

  By \cref{thm:delta-x-s-bound} and the fact that \(\mat{D}\) and \(\mat{D}^{-1}\) are diagonal matrices the search direction determined in \cref{line:compute-approximate-newton} satisfies
  \[ \norm{\hat{\Delta\vek{x}} \circ \hat{\Delta\vek{s}}}_\infty \leq \norm{\hat{\Delta\vek{x}} \circ \hat{\Delta\vek{s}}}_2 \leq \norm{\mat{D}^{-1}\hat{\Delta\vek{x}}}_2 \norm{\mat{D}\hat{\Delta\vek{s}}}_2 \leq C_1 n^2 \mu^k\]
  in every iteration, such that by \cref{thm:alpha-bar-bound} (\(\norm{\vek{v}}_\infty \leq \norm{\vek{v}}_2 \leq \gamma \sigma \mu^k/4\))
  \[ \bar{\alpha} \geq \min \Set{1, \frac{\min \Set{\gamma \sigma, 1 - \frac{5}{4}\sigma} \mu^k}{4 C_1 n^2 \mu^k}} \geq \frac{C_2}{n^2} \]
  where \(C_1, C_2 > 0\) are constants depending only on \(\gamma\) and \(\sigma\).
  Plugging this inequality in the second bound in \cref{thm:alpha-bar-bound} we get
  \[ \mu^{k+1} = \mu(\bar{\alpha}) \leq \Paren{ 1 - \Paren{1 - \frac{5}{4} \sigma} \frac{\bar{\alpha}}{2}} \mu^k \leq \Paren{ 1 - \Paren{1 - \frac{5}{4} \sigma} \frac{C_2}{2 n^2}} \mu^k \leq \Paren{1 - \frac{C_3}{n^2}} \mu^k \]
  for some constant \(C_3 > 0\).
  By induction this means that \(\mu^k \leq {(1 - \frac{C_3}{n^2})}^k \mu^0\) for all \(k \geq 0\).
  Now, let \(k \geq n^2 \log(1/\e_\mu) / C_3\) which implies
  \[ k \log\Paren{1 - \frac{C_3}{n^2}} \leq k \Paren{- \frac{C_3}{n^2}} \leq -\log(1/\e_\mu) = \log(\e_\mu) \]
  using \(\log(\beta) \leq \beta-1\) for all \(\beta > 0\).
  Applying the exponential function on both sides gives \({(1 - \frac{C_3}{n^2})}^k \leq \e_\mu\) which shows that \(O(n^2 \log(1/\e_\mu))\) iterations suffice to get \(\mu^k \leq \e_\mu \mu^0\).
\end{proof}

\section{Convergence Proof for the Approximate Newton Direction Algorithm}

Next, we turn to the correctness of \cref{alg:newton-direction} which is concerned with calculating a sufficiently good approximation of the Newton direction.
As linear convergence of the CG method was already established in \cref{thm:cg-residual-bound} our main goal is to bound the norm of the initial residual \(\mat{R}^{-T}\vek{p}\).
To this end, we introduce the following bound that can already be found in~\cite{Monteiro-ConvergenceAnalysisLongStepInfeasibleIPMs}.

\begin{lemma}[Lemma 3.2 in~\cite{Monteiro-ConvergenceAnalysisLongStepInfeasibleIPMs}, Lemma 11 in~\cite{Avron-FasterRandomizedInfeasibleIPMs}]\label{thm:eta-bound}
  Assume that
  \begin{itemize}
    \item \((\vek{x}^0, \vek{y}^0, \vek{s}^0) \in \mathcal{G}\) satisfies \((\vek{x}^0, \vek{s}^0) \geq (\vek{x}^*, \vek{s}^*)\) for some \((\vek{x}^*, \vek{y}^*, \vek{s}^*) \in \mathcal{F}^*\) and
    \item \((\vek{x}, \vek{y}, \vek{s}) \in \mathcal{G}\) satifies \(\vek{r} = \eta \vek{r}^0\) for some \(\eta \in [0, 1]\).
  \end{itemize}
  Then \(\eta \leq \vek{x}^T \vek{s} / {\vek{x}^0}^T \vek{s}^0\) implies \(\eta ({\vek{x}^0}^T \vek{s} + \vek{x}^T {\vek{s}^0}) \leq 3 n \mu\).
  % Assume that the point \((\vek{x}^0, \vek{y}^0, \vek{s}^0) \in \mathcal{G}\) satisfies \((\vek{x}^0, \vek{s}^0) \geq (\vek{x}^*, \vek{s}^*)\) for some \((\vek{x}^*, \vek{y}^*, \vek{s}^*) \in \mathcal{F}^*\).
  % Then for any point \((\vek{x}, \vek{y}, \vek{s}) \in \mathcal{G}\) such that the corresponding residuals satisfy \(\vek{r} = \eta \vek{r}^0\) for some \(\eta \in [0, 1]\) and \(\eta \leq \vek{x}^T \vek{s} / {\vek{x}^0}^T \vek{s}^0\) we have
  % \(\eta ({\vek{x}^0}^T \vek{s} + \vek{x}^T {\vek{s}^0}) \leq 3 n \mu\).
\end{lemma}

Again, this makes use of the observation that the residual \(\vek{r}\) of any iterate lies on the line segment between \(\vek{r}^0\) and \(\vek{0}\).
Note that additionally, by the definition of the neighbourhood \(\mathcal{N}_{-\infty}(\gamma)\), any point in it with \(\vek{r} = \eta \vek{r}^0\) also satisifies \(\eta \leq \mu^k / \mu^0 = {\vek{x}^k}^T \vek{s}^k / {\vek{x}^0}^T \vek{s}^0\).

\begin{lemma}[Lemma 12 in~\cite{Avron-FasterRandomizedInfeasibleIPMs} (simplified)]\label{thm:initial-residual-bound}
  Assume that
  \begin{itemize}
    \item \(\gamma \in (0, 1)\), \(\sigma \in (0, 1)\), \((\vek{x}^*, \vek{y}^*, \vek{s}^*) \in \mathcal{F}^*\),
    \item \((\vek{x}^0, \vek{y}^0, \vek{s}^0) \in \mathcal{G}\) satisfies \(\vek{x}^0 \geq \vek{x}^*\) and \(\vek{s}^0 \geq \max\set{\vek{s}^*, \abs{\mat{A}^T\vek{y}^0 - \vek{c}}}\)
    \item \((\vek{x}, \vek{y}, \vek{s}) \in \mathcal{N}_{-\infty}(\gamma)\) satisfies \(\vek{r} = \eta \vek{r}^0\) for some \(\eta \in [0, 1]\) and
    \item \(\mat{R}^{-T} \in \R^{m \times m}\) satisifies \(\norm{\mat{R}^{-T}\mat{A}\mat{D}}_2 \leq \sqrt{2}\).
  \end{itemize}
  Then
  \[ \norm{\mat{R}^{-T} \vek{p}}_2 \leq \sqrt{\frac{2 \mu}{1 - \gamma}} \Paren{7n + \sigma \sqrt{n}} \leq 8 \sqrt{\frac{2}{1 - \gamma}} n \sqrt{\mu} \]
  where \(\vek{p} = -\vek{r}_p + \mat{A} ( -\mat{S}^{-1} \mat{X} \vek{r}_d + \vek{x} - \sigma \mu \mat{S}^{-1} \vek{1} )\) as in \cref{eqn:normal-equation}.
\end{lemma}

\begin{proof}
  First, we will deduce bounds for \(\mat{S}(\vek{x}^0 - \vek{x}^*)\) and \(\mat{X}\vek{r}_d^0\) that we need later.
  Note that because \(\vek{0} \leq \vek{x}^* \leq \vek{x}^0\) and \(\vek{s} \geq \vek{0}\) we have \(\vek{0} \leq \mat{S}(\vek{x}^0 - \vek{x}^*) \leq \mat{S}\vek{x}^0 \), so
  \[
    \norm{\mat{S}(\vek{x}^0 - \vek{x}^*)}_2
    \leq \norm{\mat{S}\vek{x}^0}_2
    \leq \norm{\mat{S}\vek{x}^0}_1
    = \vek{s}^T \vek{x}^0.
  \]
  Using a similar argument as above, \(\vek{0} \leq \abs{\mat{A}^T \vek{y}^0 - \vek{c}} \leq \vek{s}^0\) implies \(\norm{\mat{X} \abs{\mat{A}^T \vek{y}^0 - \vek{c}}}_2 \leq \norm{\mat{X}\vek{s}^0}_2\) and therefore
  \begin{align*}
    \norm{\mat{X}\vek{r}_d^0}_2
    &= \norm{\mat{X}(\mat{A}^T \vek{y}^0 + \vek{s}^0 - \vek{c})}_2
    \leq \norm{\mat{X}\vek{s}^0}_2 + \norm{\mat{X} \abs{\mat{A}^T \vek{y}^0 - \vek{c}}}_2 \\
    &\leq 2 \norm{\mat{X}\vek{s}^0}_2
    \leq 2 \norm{\mat{X}\vek{s}^0}_1
    = 2 \vek{x}^T \vek{s}^0.
  \end{align*}
  Now, consider the expression in the claim. Because
  \begin{align*}
    \mat{R}^{-T}\vek{p}
    &= \mat{R}^{-T}\Paren{-\vek{r}_p + \mat{A} ( -\mat{S}^{-1} \mat{X} \vek{r}_d + \vek{x} - \sigma \mu \mat{S}^{-1} \vek{1} )} \\
    &= \mat{R}^{-T}\Paren{-\vek{r}_p -\mat{A}\mat{S}^{-1}\mat{X} \vek{r}_d + \mat{A}\vek{x} - \sigma \mu \mat{A}\mat{S}^{-1} \vek{1} } \\
    &= \mat{R}^{-T}\Paren{-\eta \vek{r}_p^0 - \eta \mat{A}\mat{S}^{-1}\mat{X} \vek{r}_d^0 + \mat{A}\vek{x} - \sigma \mu \mat{A}\mat{S}^{-1} \vek{1} } \\
    &= \mat{R}^{-T}\Paren{- \eta \mat{A}(\vek{x}^0 - \vek{x}^*) - \eta \mat{A}\mat{S}^{-1}\mat{X} \vek{r}_d^0 + \mat{A}\vek{x} - \sigma \mu \mat{A}\mat{S}^{-1} \vek{1} } \\
    &= \mat{R}^{-T}\mat{A}\Paren{- \eta (\vek{x}^0 - \vek{x}^*) - \eta \mat{S}^{-1}\mat{X} \vek{r}_d^0 + \vek{x} - \sigma \mu \mat{S}^{-1} \vek{1} } \\
    &= \mat{R}^{-T}\mat{A}\mat{S}^{-1}\Paren{- \eta \mat{S}(\vek{x}^0 - \vek{x}^*) - \eta \mat{X} \vek{r}_d^0 + \mat{S}\vek{x} - \sigma \mu \vek{1} } \\
    &= \mat{R}^{-T}\mat{A}\mat{D}{(\mat{S}\mat{X})}^{-1/2}\Paren{- \eta \mat{S}(\vek{x}^0 - \vek{x}^*) - \eta \mat{X} \vek{r}_d^0 + \mat{S}\vek{x} - \sigma \mu \vek{1} }
  \end{align*}
  using the definition of \(\vek{p}\) from \cref{eqn:normal-equation} and the assumption \(\vek{r} = \eta \vek{r}^0\), it suffices to bound each part of the last expression.
  \(\norm{\mat{R}^{-T}\mat{A}\mat{D}}_2 \leq \sqrt{2}\) by assumption,
  \[\norm{{(\mat{S}\mat{X})}^{-1/2}}_2 = \max_{1 \leq i \leq n} {(x_i s_i)}^{-1/2} \leq \sqrt{\frac{1}{(1 - \gamma) \mu}}\]
  by the definition of the neighbourhood and
  \(\norm{\mat{S}\vek{x}}_2 \leq \norm{\mat{S}\vek{x}}_1 = n \mu\)
  using the definition of \(\mu\).
  \(\norm{\mat{S}(\vek{x}^0 - \vek{x}^*)}_2\) and \(\norm{\mat{X}\vek{r}_d^0}_2\) were already bounded above and lastly, \(\norm{\vek{1}}_2 = \sqrt{n}\).
  This implies
  \begin{align*}
    \norm{\mat{R}^{-T} \vek{p}}_2 
    &\leq \norm{\mat{R}^{-T}\mat{A}\mat{D}}_2 \norm{{(\mat{S}\mat{X})}^{-1/2}}_2 \Paren{\eta \norm{\mat{S}(\vek{x}^0 - \vek{x}^*)} + \eta \norm{\mat{X}\vek{r}_d^0} + \norm{\mat{S}\vek{x}}_2 + \sigma \mu \norm{\vek{1}}_2} \\
    &\leq \sqrt{2} \sqrt{\frac{1}{(1-\gamma) \mu}} \Paren{\eta \vek{s}^T \vek{x}^0 + 2 \eta \vek{x}^T \vek{s}^0 + n\mu + \sigma \mu \sqrt{n}} \\
    &\leq \sqrt{2} \sqrt{\frac{1}{(1-\gamma) \mu}} \Paren{6 n \mu + n\mu + \sigma \mu \sqrt{n}}
    = \sqrt{\frac{2 \mu}{1 - \gamma}} \Paren{7 n + \sigma \sqrt{n}}
  \end{align*}
  with the help of \cref{thm:eta-bound}.
\end{proof}

Equipped with this result, we can prove the main result regarding \cref{alg:newton-direction}.

\begin{proof}[Proof of \cref{thm:approximate-newton-convergence}]
  By \cref{thm:sparse-ose} the matrix \(\mat{W}\) selected in \cref{line:draw-sketching-matrix} of \cref{alg:newton-direction} is a subspace embedding for \(\mat{D}\mat{A}^T\) with probability at least \(1 - \delta\), so it suffices to show all of the claims for this case.
  Therefore, assume that \(\mat{W}\) is indeed a \((1 \pm \e)\) \(\ell_2\)-subspace embedding for \(\mat{D}\mat{A}^T\) from now on.
  In this case the QR decomposition \(\mat{Q}\mat{R} = \mat{W}\mat{D}\mat{A}^T\) exists and by \cref{thm:condition-number-bound} we have
  \[\norm{\mat{R}^{-T}\mat{A}\mat{D}}_2 \leq \sqrt{\frac{1}{1 - \e}} \leq \sqrt{2}\]
  using \(\e < \frac{1}{2}\).
  Let \(\vek{q}^j\) be the iterates of the CG algorithm in \cref{line:cg} and let \(\vek{f}^j\) be the residuals defined by
\( \vek{f}^j = \mat{R}^{-T} \mat{A} \mat{D}^2 \mat{A} \mat{R}^{-1} \vek{q}^j - \mat{R}^{-T} \vek{p} \).
  Using \cref{thm:initial-residual-bound} and \(\vek{q}^0 = \vek{0}\) we have \(\norm{\vek{f}^0}_2 = \norm{\mat{R}^{-T} \vek{p}}_2 \leq Cn\sqrt{\mu}\) for some constant \(C > 0\) depending on \(\gamma\).
  \Cref{thm:cg-residual-bound} immediately implies that \(\norm{\vek{f}^j}_2\) converges to zero and the so \cref{line:cg} will terminate.
  Moreover, it tells us that for \(j \geq \log_{\frac{1 - \e}{1 + \e}}(\e_{\vek{v}}/C n^2 \mu) = \log_{\frac{1 + \e}{1 - \e}}(C n^2 \mu / \e_{\vek{v}})\) we have
  \[ \norm{\vek{f}^j}_2 \leq \Paren{\frac{1 - \e}{1 + \e}}^j \norm{\vek{f}^0}_2 \leq \frac{\e_{\vek{v}}}{C n^2 \mu} C n \sqrt{\mu} \leq \frac{\e_{\vek{v}}}{n \sqrt{\mu}} \]
  so \cref{line:cg} will terminate after \(O(\log(n^2 \mu / \e_{\vek{v}}))\) CG iterations.
  By \cref{thm:perturbation-vector} and the definition of \(\vek{v}\) in \cref{line:compute-v} \(\norm{\vek{f}^j}_2 \leq \e_{\vek{v}}/n\sqrt{\mu}\) implies \(\norm{\vek{v}}_2 \leq \e_{\vek{v}}\).
  \Cref{line:compute-delta-x,line:compute-delta-y,line:compute-delta-s} compute \((\hat{\Delta\vek{x}}, \hat{\Delta\vek{y}}, \hat{\Delta\vek{s}})\) which satisfy \cref{eqn:perturbed-search-direction-choice} for \(\tilde{\vek{f}} = \mat{R}^T \vek{f}\).
  Using \cref{thm:perturbation-vector} again we can see that
  \begin{align*}
    \mat{A}\hat{\Delta\vek{x}} 
    &= \mat{A} (-\vek{x} + \sigma \mu \mat{S}^{-1}\vek{1} - \mat{S}^{-1}\mat{X}\hat{\Delta\vek{s}} - \mat{S}^{-1}\vek{v}) \\
    &= -\mat{A}\vek{x} + \sigma \mu \mat{A}\mat{S}^{-1}\vek{1} - \mat{A}\mat{S}^{-1}\mat{X}\hat{\Delta\vek{s}} - \mat{A}\mat{S}^{-1}\vek{v} \\
    &= -\mat{A}\vek{x} + \sigma \mu \mat{A}\mat{S}^{-1}\vek{1} + \mat{A}\mat{S}^{-1}\mat{X}\vek{r}_d + \mat{A}\mat{S}^{-1}\mat{X}\mat{A}^T \hat{\Delta\vek{y}} - \mat{A}\mat{S}^{-1}\vek{v} \\
    &= -\mat{A}\vek{x} + \sigma \mu \mat{A}\mat{S}^{-1}\vek{1} + \mat{A}\mat{S}^{-1}\mat{X}\vek{r}_d + \vek{p} + \vek{R}^T\vek{f} - \mat{A}\mat{S}^{-1}\vek{v} \\
    &= -\vek{r}_p + \mat{R}^T \vek{f} - \mat{A}\mat{S}^{-1}\vek{v} \\
    &= -\vek{r}_p \\
    \mat{A}^T \hat{\Delta\vek{y}} + \hat{\Delta\vek{s}}
    &= \mat{A}^T \hat{\Delta\vek{y}} - \vek{r}_d - \mat{A}^T \hat{\Delta\vek{y}} \\
    &= -\vek{r}_d \\
    \mat{S}\hat{\Delta\vek{x}} + \mat{X}\hat{\Delta\vek{s}} 
    &= \mat{S}(-\vek{x} + \sigma \mu \mat{S}^{-1}\vek{1} - \mat{S}^{-1}\mat{X}\hat{\Delta\vek{s}} - \mat{S}^{-1}\vek{v}) + \mat{X}\hat{\Delta\vek{s}} \\
    &= -\mat{S}\vek{x} + \sigma \mu \vek{1} - \mat{X}\hat{\Delta\vek{s}} - \vek{v} + \mat{X}\hat{\Delta\vek{s}} \\
    &= -\vek{x} \circ \vek{s} + \sigma \mu \vek{1} - \vek{v}
  \end{align*}
  which shows that the determined search direction indeed satisfies \cref{eqn:perturbed-search-direction} with \(\norm{\vek{v}}_2 \leq \e_{\vek{v}}\).

  After showing that the algorithm works correctly, we need to bound the running time.
  Multiplication with \(\mat{X}, \mat{X}^{1/2}, \mat{S}, \mat{S}^{1/2}\) and their inverses can be done in \(O(n)\), multiplication with \(\mat{A}\) and \(\mat{A}^T\) in \(O(\nnz(\mat{A}))\) and multiplication with \(\mat{R}^{-1}\) and \(\mat{R}^{-T}\) in \(O(m^2)\) using triangular solves.
  Because \(m \leq n \leq \nnz(\mat{A})\) we can thus conclude that \cref{line:compute-residuals,line:compute-p,line:compute-delta-x,line:compute-delta-y,line:compute-delta-s} run in \(O(\nnz(\mat{A}) + m^2)\) operations.
  Constructing the random sparse embedding matrix in \cref{line:draw-sketching-matrix} requires us to choose \(n \cdot s\) random nonzero entries and therefore needs \(O(n s)\) time.
  Computing \(\mat{W}\mat{D}\mat{A}^T\) in \cref{line:compute-qr} can be done in \(O(s \nnz(\mat{A}))\) by taking the product of each nonzero entry of \(\mat{D}\mat{A}^T\) with each nonzero entry in the appropriate column of \(\mat{W}\) and adding the result to the appropriate entry in the matrix product.
  Computing the QR decomposition of a \(w \times m\) matrix takes \(O(w m^2)\) operations.
  The number of CG iterations in \cref{line:cg} were already bounded by \(O(\log(n^2 \mu / \e_{\vek{v}}))\) and the runtime of each iteration is dominated by a matrix-vector product involving \(\mat{R}^{-T}\mat{A}\mat{D}^2\mat{A}^T\mat{R}^{-1}\) which takes \(O(\nnz(\mat{A}) + m^2)\) flops.
  Lastly, the perturbation vector \(\vek{v}\) is calculated in \cref{line:compute-v} in \(O(ns + m^2)\) time.
  Combining these upper bounds gives
  \[ O \Paren{\nnz(\mat{A}) + m^2 + ns + s \nnz(\mat{A}) + wm^2 + \log \Paren{n^2 \mu / \e_{\vek{v}}} \Paren{\nnz(\mat{A}) + m^2}} \]
  operations which simplifies to
  \[ O \Paren{\log \Paren{m/\delta} \Paren{\nnz(\mat{A}) + m^3} + \log \Paren{n^2 \mu / \e_{\vek{v}}} \Paren{\nnz(\mat{A}) + m^2}} \]
  operations because \(w\) and \(s\) were chosen according to \cref{thm:sparse-ose}.
\end{proof}

% \begin{remark}
%   A crucial observation at this point is that the QR decomposition of the sketched matrix \(\mat{W}\mat{D}\mat{A}^T\) needs \(O(wm^2)\) operations and is therefore asymptotically at least as expensive as a Cholesky decomposition of \(\mat{A}\mat{D}^2\mat{A}^T\) which takes \(O(m^3)\) operations.
% \end{remark}

\section{Runtime Bounds for the Combined Algorithm}

\Cref{thm:ipm-convergence,thm:approximate-newton-convergence} were the stepping stones towards the final result of this chapter.
The combination of \cref{alg:ipm,alg:newton-direction} is an efficient algorithm for solving linear programs where \(\mat{A}\) is sparse and \(m \ll n\):

\begin{theorem}\label{thm:combined-algorithm-convergence}
  Assume that \(\gamma \in (0, 1)\), \(\sigma \in (0, \frac{4}{5})\), \(\e \in (0, \frac{1}{2})\) and \(\e_\mu \in (0, 1)\) are constant, that the initial point \((\vek{x}^0, \vek{y}^0, \vek{s}^0) \in \mathcal{G}\) satisfies \(\vek{x}^0 \circ \vek{s}^0 \geq (1-\gamma)\vek{1}\), \(\vek{x}^0 \geq \vek{x}^*\) and \(\vek{s}^0 \geq \max \{\vek{s}^*, \abs{\mat{A}^T \vek{y}^0 - \vek{c}}\}\) for some \((\vek{x}^*, \vek{y}^*, \vek{s}^*) \in \mathcal{F}^*\) and that \(\mat{A}\) has full row rank.
  Furthermore, assume that \cref{alg:ipm} uses \cref{alg:newton-direction} with \(\e_{\vek{v}} = \gamma \sigma \mu^k / 4\) to determine the approximate Newton direction in \cref{line:compute-approximate-newton,line:prepare-approximate-newton}.
  Then it is possible to choose \(\delta\) in such a way that the algorithm generates an iterate \((\vek{x}^k, \vek{y}^k, \vek{s}^k)\) satisfying \(\mu^k \leq \e_\mu \mu^0\) and \(\norm{\vek{r}^k} \leq \e_\mu \norm{\vek{r}^0}\) using \(O(n^2 \log(n) (\nnz(\mat{A}) + m^3))\) operations with probability at least \(99\%\).
\end{theorem}
\begin{proof}
  The assumptions of this theorem and the invariants \((\vek{x}^k, \vek{y}^k, \vek{s}^k) \in \mathcal{N}_{-\infty}(\gamma)\) and \(\vek{r}^k = \eta \vek{r}^0\) for some \(\eta \in [0, 1]\) provided by \cref{alg:ipm} provide the assumptions of \cref{thm:ipm-convergence} and \cref{thm:approximate-newton-convergence}, so the combined algorithm will generate an iterate \((\vek{x}^k, \vek{y}^k, \vek{s}^k)\) satisfying \(\mu^k \leq \e_\mu \mu^0\) and \(\norm{\vek{r}^k} \leq \e_\mu \norm{\vek{r}^0}\) after \(O(n^2 \log(1/e_\mu))\) IPM iterations as long as \cref{alg:newton-direction} does not fail in any iteration.
  To simplify the runtime bounds we assume that \(\e_\mu\) is constant, so \(O(n^2 \log(1/\e_\mu)) = O(n^2)\).

  The probability of zero failures over \(k\) IPM iterations is at least \({(1 - \delta)}^k\) because the random sketching matrices are independently drawn for each iteration.
  As the overall failure probability should be below some arbitrary constant, say \(\delta_{\mathrm{IPM}} = 1\%\), and the number of IPM iterations is bounded by \(C n^2\) for some constant \(C > 0\), it suffices to choose \(\delta\) small enough such that
  \( {(1 - \delta)}^{C n^2} \geq 1 - \delta_{\mathrm{IPM}} \).
  For 
  \[ \delta = -\frac{\log(1 - \delta_{\mathrm{IPM}} / 2)}{Cn^2} = O(n^{-2})\]
  we get
  \[ \lim_{n \to \infty} {(1 - \delta)}^{C n^2} = \lim_{n \to \infty} \Paren{1 + \frac{\log(1 - \delta_{\mathrm{IPM}} / 2)}{Cn^2}}^{C n^2} = 1 - \delta_{\mathrm{IPM}} / 2 > 1 - \delta_{\mathrm{IPM}} \]
  so \( {(1 - \delta)}^{C n^2} \geq 1 - \delta_{\mathrm{IPM}} \) is satisfied for \(n\) large enough.
  In other words, for large enough instances an arbitrarily low fixed failure probability \(\delta_{\mathrm{IPM}} > 0\) can be guaranteed for some \(\delta = O(n^{-2})\).

  With this choice of \(\delta\) and \(\e_{\vek{v}} = \gamma \sigma \mu / 4\) \cref{line:prepare-approximate-newton,line:compute-approximate-newton} of \cref{alg:ipm} (which use \cref{alg:newton-direction}) need
  \begin{align*}
    &\phantom{{}={}} O \Paren{\log \Paren{m/\delta} \Paren{\nnz(\mat{A}) + m^3} + \log \Paren{n^2 \mu / \e_{\vek{v}}} \Paren{\nnz(\mat{A}) + m^2}} \\
    &= O \Paren{\log \Paren{mn^2} \Paren{\nnz(\mat{A}) + m^3} + \log \Paren{n^2} \Paren{\nnz(\mat{A}) + m^2}} \\
    &= O \Paren{\log(n) \Paren{\nnz(\mat{A}) + m^3}}
  \end{align*}
  operations per iteration by \cref{thm:approximate-newton-convergence}.
  Note that asymptotically the computation of \(\mat{W}\mat{D}\mat{A}^T\) and its decomposition are among the most expensive operations.

  To show that the remaining operations in \cref{line:alpha-tilde,line:alpha-bar,line:compute-next-iterate,line:increment-k} do not change the asymptotic runtime bound for each IPM iteration, first consider
  \begin{align*}
    \vek{x}^k(\alpha) &\coloneqq \vek{x}^k + \alpha \hat{\Delta\vek{x}} \\
    \vek{y}^k(\alpha) &\coloneqq \vek{y}^k + \alpha \hat{\Delta\vek{y}} \\
    \vek{s}^k(\alpha) &\coloneqq \vek{s}^k + \alpha \hat{\Delta\vek{s}} \\
    \mu^k(\alpha) &\coloneqq {\vek{x}^k(\alpha)}^T \vek{s}^k(\alpha) \\
    \vek{r}^k(\alpha) &\coloneqq (\mat{A}\vek{x}^k(\alpha) - \vek{b}, \mat{A}^T \vek{y}^k(\alpha) + \vek{s}^k(\alpha) - \vek{c}).
  \end{align*}
  and their dependence on \(\alpha\).
  As \(\vek{x}^k(\alpha)\), \(\vek{y}^k(\alpha)\) and \(\vek{s}^k(\alpha)\) are linear in \(\alpha\), \(\mu^k(\alpha)\) and \(\vek{x}^k(\alpha) \circ \vek{s}^k(\alpha)\) are quadratic functions in alpha in each component.
  Because the search direction satisfies \cref{eqn:perturbed-search-direction}, \(\vek{r}^k(\alpha) = (1 - \alpha)\vek{r}^k(0)\) and so \(\norm{\vek{r}^k(\alpha)}_2 / \norm{\vek{r}^0}_2\) is linear in \(\alpha\).
  Therefore all terms of interest are either linear or quadratic in \(\alpha\) and we will assume that all polynomial coefficients for these functions are determined at the start which can be done in \(O(n)\).

  We begin in \cref{line:alpha-tilde} in which the maximum \(\tilde{\alpha} \in [0, 1]\) is determined such that for all \(\alpha \in [0, \tilde{\alpha}]\) we have \((\vek{x}^k, \vek{y}^k, \vek{s}^k) + \alpha (\hat{\Delta\vek{x}}, \hat{\Delta\vek{y}}, \hat{\Delta\vek{s}}) \in \mathcal{N}_{-\infty}(\gamma)\).
  Define
  \begin{align*}
    \tilde{\alpha}_1 &= \max \set{\alpha' \in [0, 1] | \vek{x}^k(\alpha) \circ \vek{s}^k(\alpha) \geq (1 - \gamma)\mu^k(\alpha) \vek{1} \ \forall \alpha \in [0, \alpha']} \\
    \tilde{\alpha}_2 &= \max \Set{\alpha' \in [0, 1] | \frac{\norm{\vek{r}^k(\alpha)}_2}{\norm{\vek{r}^0}_2} \leq \frac{\mu^k(\alpha)}{\mu^0} \ \forall \alpha \in [0, \alpha']}
  \end{align*}
  and note that
  \begin{align*}
    \tilde{\alpha}_1 &= \min \set{ \alpha \in [0, 1] | x_i^k(\alpha) s_i^k(\alpha) - (1 - \gamma)\mu^k(\alpha) = 0 \text{ for some } 1 \leq i \leq n } \\
    \tilde{\alpha}_2 &= \min \Set{ \alpha \in [0, 1] | \frac{\norm{\vek{r}^k(\alpha)}_2}{\norm{\vek{r}^0}_2} - \frac{\mu^k(\alpha)}{\mu^0} = 0 }
  \end{align*}
  because \((\vek{x}^k(0), \vek{y}^k(0), \vek{s}^k(0)) \in \mathcal{N}_{-\infty}(\gamma)\).
  Using the second characterisation \(\tilde{\alpha}_1\) can be determined in \(O(n)\) by solving \(n\) quadratic equations and \(\tilde{\alpha}_2\) can be determined in \(O(1)\) by solving one quadratic equation.

  At this point \(\tilde{\alpha} = \min\{\tilde{\alpha}_1, \tilde{\alpha}_2\}\) even though the condition \(\vek{x}^k(\alpha), \vek{s}^k(\alpha) > \vek{0}\) which is part of the defintion of \(\mathcal{N}_{-\infty}(\gamma)\) is missing:
  Any negative value of \(x_i(\alpha)\) or \(s_i(\alpha)\) is preceded by a zero of \(x_i(\alpha) s_i(\alpha) - (1 - \gamma)\mu(\alpha)\), so because \(\tilde{\alpha} \geq \tilde{\alpha}_1\) all entries of \(\vek{x}^k(\tilde{\alpha})\) and \(\vek{s}^k(\tilde{\alpha})\) are nonnegative.
  If \(x_i(\tilde{\alpha}) = 0\) or \(s_i(\tilde{\alpha}) = 0\) for some \(i\) then \(\vek{x}(\tilde{\alpha}) \circ \vek{s}(\tilde{\alpha}) \geq (1 - \gamma)\mu(\tilde{\alpha}) \vek{1}\) implies \(\mu(\tilde{\alpha}) = 0\) in which case we also have \(\vek{r}(\tilde{\alpha}) = 0\).
  This means that the only case where \(\vek{x}(\tilde{\alpha}), \vek{s}(\tilde{\alpha}) > \vek{0}\) might be violated is when a solution was determined and in this case the algorithm will terminate.

  Next, computing \(\bar{\alpha}\) in \cref{line:alpha-bar} is equivalent to minimizing the quadratic function \(\mu(\alpha)\) in \([0, \tilde{\alpha}]\).
  Computing the stationary point of the quadratic function \(\mu(\alpha)\) takes \(O(1)\) and determining whether the minimum is attained at the boundary of \([0, \tilde{\alpha}]\) or at the stationary point can also be done in \(O(1)\).

  Lastly, \cref{line:compute-next-iterate} takes \(O(n)\) and \cref{line:increment-k} \(O(1)\) operations.
  We can conclude that all computations outside of \cref{line:prepare-approximate-newton,line:compute-approximate-newton} can be completed in \(O(n)\) and therefore the worst-case complexity per iteration is still bounded by \( O \Paren{\log(n) \Paren{\nnz(\mat{A}) + m^3}} \).
  Because \(O(n^2)\) iterations need to be performed the total complexity is
  \[ O \Paren{n^2 \log(n) \Paren{\nnz(\mat{A}) + m^3}} \]
  as claimed.
\end{proof}
