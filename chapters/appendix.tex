\chapter{Source Code}
%TODO: Rename this chapter

This appendix contains the key elements of the implementation concerned with generating the sketching matrix and the preconditioning process.

\begin{minted}[breaklines]{python}
import numpy as np
import scipy

...

default_rng = np.random.default_rng()

def _construct_sketching_matrix(
    w, n, s=3, sparse=True, rng=default_rng
):
    """
    Randomly construct a sketching matrix of size (w, n) where w is
    assumed to be smaller than n. If sparse is True, each column
    contains s nonzero entries randomly drawn from +-1/sqrt(s).
    Otherwise all matrix entries are drawn from a normal distribution.

    Parameters
    ----------
    w : int
        Number of rows of the sketching matrix
    n : int
        Number columns of the sketching matrix
    s : int
        Number of nonzero entries in each row if ``sparse = True``
    sparse : bool
        True if the sketching matrix should be sparse.
    rng : np.random.Generator (default = np.random.default_rng())
        A random number generator used to construct the random matrix

    Returns
    -------
    A random sketching matrix

    """
    if sparse:
        # The matrix is constructed in the COO format:
        # data[i] is the entry at (row_indices[i], column_indices[i])
        # for each index i

        # 1. Generate the entries: randomly chosen from +- 1/sqrt(s)
        data = rng.choice(
            [-1 / np.sqrt(s), 1 / np.sqrt(s)], size=s * n
        )

        # 2. Generate the row indices: Draw s elements from
        # {1, ..., w} without replacement 1.1*n times. Delete all
        # groups of s elements where the same element was chosen twice
        # and take the first n of the remaining groups to be the
        # row_indices.
        row_indices = rng.choice(w, size=(int(n * 1.1), s))
        row_indices.sort()
        row_indices = row_indices[
            (row_indices[..., 1:] != row_indices[..., :-1]).all(
                axis=-1
            )
        ]
        row_indices = row_indices[:n]
        row_indices.shape = (n * s,)

        # 3. Generate the column indices by repeating each i from
        # {1, ..., n} s times: [1, 1, 1, 2, 2, 2, ..., n, n, n] for
        # s = 3 for example
        column_indices = np.repeat(np.arange(n), s)

        # 4. Construct the matrix as described and convert it to the
        # CSR sparse matrix format before returning to allow fast
        # matrix multiplication
        mat = scipy.sparse.coo_matrix(
            (data, (row_indices, column_indices)), shape=(w, n)
        )
        return mat.tocsr()
    else:
        # Return a Gaussian sketch
        return rng.normal(size=(w, n)) / np.sqrt(w)


# This function is called once for every IPM iteration to get a method
# for solving the normal equation for arbitrary right-hand sides. Note
# that the notation for the matrices in the paper by Andersen and
# Andersen is different the one used in this work. The matrix "Dinv"
# thus corresponds to D^2, "Dinv_half" corresponds to D and "M"
# corresponds to A D^2 A^T. The diagonal matrix "Dinv" is passed to
# this function as a 1-D vector of its diagonal elements.
def _get_solver(A, Dinv, options):
    """
    Given ``A`` and `` Dinv`` return a handle to a linear system
    solver for the matrix ``M = A @ Dinv @ A.T``, i.e. a function that
    takes in an arbitrary right-hand side ``r`` and returns
    ``M^{-1} r``

    Parameters
    ----------
    A : 2-D array
    Dinv : 1-D array
        As defined in [4] Equation 8.31
    options : AllOptions
        Provides the options to choose the linear solver.

    Returns
    -------
    solve : function
        Handle to the appropriate solver function

    References
    ----------
    .. [4] Andersen, Erling D., and Knud D. Andersen. "The MOSEK
           interior point optimizer for linear programming: an
           implementation of the homogeneous algorithm." High
           performance optimization. Springer US, 2000. 197-232.
    """
    method = options.linear_solver.solver

    if method is Solver.CHOLESKY:
        # The function scipy.sparse.diags turns a 1-D vector into a
        # sparse diagonal matrix where the vector entries are on the
        # diagonal
        M = A @ scipy.sparse.diags(Dinv, 0, format="csc") @ A.T

        # The matrix L is the Cholesky factor of M: LL^T = M
        L = scipy.linalg.cho_factor(M.toarray())

        def solve(r):
            return scipy.linalg.cho_solve(L, r)

        return solve

    elif method is Solver.QR:
        Dinv_half = np.sqrt(Dinv)
        M_half = scipy.sparse.diags(Dinv_half, format="csc") @ A.T

        # Find the QR decomposition of DA^T. The toarray() method
        # converts the sparse matrix to a dense one so that a dense QR
        # solver can be used.
        R = np.linalg.qr(M_half.toarray(), mode="r")

        def solve(r):
            r2 = scipy.linalg.solve_triangular(
                R, r, trans="T"
            )  # r2 = R^{-T} r
            x = scipy.linalg.solve_triangular(
                R, r2, trans="N"
            )  # x = R^{-1} r2
            return x

        return solve

    elif method is Solver.PRECONDITIONED_CG:
        Dinv_half = np.sqrt(Dinv)

        # Construct the sketching matrix according to the chosen
        # options:
        # - sketching_factor determines the w/m factor
        # - sketching_sparsity determines the value of s
        sketching_matrix = _construct_sketching_matrix(
            w=int(
                options.preconditioning.sketching_factor * A.shape[0]
            ),
            n=A.shape[1],
            s=options.preconditioning.sketching_sparsity,
            sparse=True,
        )

        # Explicitly construct the sketched matrix WDA^T ...
        sketched_matrix = (
            sketching_matrix
            @ scipy.sparse.diags(Dinv_half, format="csc")
            @ A.T
        )

        # ... and find its QR decomposition
        R = np.linalg.qr(sketched_matrix.toarray(), mode="r")

        # scipy.sparse.linalg.aslinearoperator turns matrices into
        # linear operators. They can be chained together using matrix
        # multiplication syntax without forming the product
        # explicitly. Taking the product with a vector afterwards
        # applies all of the matrices in the right order.
        A_op = scipy.sparse.linalg.aslinearoperator(A)
        Dinv_op = scipy.sparse.linalg.aslinearoperator(
            scipy.sparse.diags(Dinv, 0, format="csc")
        )
        # The linear operator of R^{-1} is constructed by defining
        # that multiplication with R^{-1} and R^{-T} can be performed
        # using triangular solves
        Rinv_op = scipy.sparse.linalg.LinearOperator(
            R.shape,  # R is square
            matvec=lambda x: scipy.linalg.solve_triangular(
                R, x, trans="N"
            ),
            rmatvec=lambda x: scipy.linalg.solve_triangular(
                R, x, trans="T"
            ),
        )
        # The preconditioned matrix is now given by the correct chain
        # of matrix-vector multiplications
        preconditioned_matrix = (
            Rinv_op.T @ A_op @ Dinv_op @ A_op.T @ Rinv_op
        )

        def solve(r):
            # Solve the preconditioned linear system with the CG
            # method:
            # - maxiter bounds the number of CG iterations
            # - tol is the required relative tolerance (2-norm)
            # - atol is the required absolute tolerance (2-norm)
            # We used tol = atol = 0; maxiter = 50, 75, 100, 125
            x, info = scipy.sparse.linalg.cg(
                A=preconditioned_matrix,
                b=Rinv_op.T @ r,
                maxiter=options.linear_solver.solver_maxiter,
                tol=options.linear_solver.solver_rtol,
                atol=options.linear_solver.solver_atol,
            )
            # info < 0 indicates a numerical failure
            if info < 0:
                raise scipy.linalg.LinAlgError(f"CG failed ({info})!")
            return Rinv_op @ x

        return solve
\end{minted}