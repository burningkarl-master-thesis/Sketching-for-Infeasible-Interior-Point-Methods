\chapter{Sketching --- A Useful Dimensionality Reduction Technique}\label{chap:sketching}

Based on~\cite{Avron-FasterRandomizedInfeasibleIPMs} we will use sketching to determine a preconditioner for \(\mat{A} \mat{D}^2 \mat{A}^T\).
These results not only apply to this specific matrix but all symmetric positive definite matrices of the form \(\mat{B}^T \mat{B}\) for some \(\mat{B} \in \R^{n \times m}\) with \(m \ll n\).
To emphasise this, the theorems will all involve a general matrix \(\mat{B}\) and their consequences for \(\mat{A}\mat{D}^2\mat{A}^T\) will be discussed afterwards.

\section{Overview and a Sparse Sketching Result}\label{sec:sketching-overview}

The following introduction to sketching is based on~\cite{Woodruff-Sketching} and begins with the central concept of a subspace embedding:
\begin{definition}\label{def:subspace-embedding}
Let \(\e \in (0, 1)\). A \emph{\((1 \pm \e)\) \(\ell_2\)-subspace embedding} for the column space of a matrix \(\mat{B} \in \R^{n \times m}\) is a matrix \(\mat{W} \in \R^{w \times n}\) such that
\[ (1-\e) \norm{\mat{B}\vek{z}}_2^2 \leq \norm{\mat{W}\mat{B}\vek{z}}_2^2 \leq (1+\e)\norm{\mat{B}\vek{z}}_2^2 \]
for all \(\vek{z} \in \R^m\).
\end{definition}
Note that this definition does not depend on a particular basis for the representation of the column space of \(\mat{B}\).
Nonetheless, we will often refer to \(\mat{W}\) as a \((1\pm\e)\) \(\ell_2\)-subspace embedding of \(\mat{B}\).
Intuitively, a subspace embedding of some \(m\)-dimensional subspace of \(\R^n\) is a linear map from the (large) space \(\R^n\) to a (smaller) space \(\R^w\) that approximately preserves all of the distances of the subspace.
There are multiple equivalent characterisations of subspace embeddings used in the literature.

\begin{theorem}
Assume that \(\mat{B} \in \R^{n \times m}\) has full column rank and let
\[\mathcal{S} = \set{ \mat{B}\vek{z} | \vek{z} \in \R^m } = \set{ \mat{U}\vek{z} | \vek{z} \in \R^m } \subset \R^n\]
for some matrix \(\mat{U} \in \R^{n \times m}\) with orthonormal columns.
Then the following statements are equivalent
\begin{enumerate}
  \item \(\mat{W}\) is a \((1\pm\e)\) \(\ell_2\)-subspace embedding for the column space of \(\mat{B}\)
  \item \((1-\e) \norm{\mat{B}\vek{z}}_2^2 \leq \norm{\mat{W}\mat{B}\vek{z}}_2^2 \leq (1+\e)\norm{\mat{B}\vek{z}}_2^2\) for all \(\vek{z} \in \R^d\)
  \item \(\abs{\norm{\mat{W}\vek{s}}_2^2 - 1} \leq \e\) for all \(\vek{s} \in \mathcal{S}\) with \(\norm{\vek{s}}_2 = 1\)
  \item \(\norm{\mat{U}^T \mat{W}^T \mat{W} \mat{U} - \mat{I}_d}_2 \leq \e\)
\end{enumerate}
\end{theorem}
\begin{proof}
The first equivalence holds by definition. Statement 3 is equivalent to
\[ 1-\e \leq \norm{\mat{W}\mat{s}}_2^2 \leq 1+\e \quad \forall \vek{s} \in \mathcal{S} \text{ with } \norm{\vek{s}}_2 = 1. \]
In this form statement 2 immediately implies 3 and choosing \(\vek{s} = \norm{\mat{B}\vek{z}}_2^{-1} \mat{B}\vek{z}\) for \(\mat{B}\vek{z} \neq \vek{0}\) and multiplying both sides by \(\norm{\mat{B}\vek{z}}_2^2\) we obtain that the two statements are indeed equivalent.

Note that the matrix \(\mat{U}^T \mat{W}^T \mat{W} \mat{U} - \mat{I}_d\) is real and symmetric and therefore its 2-norm is given by the maximum absolute value of the Rayleigh quotient~\cite[Equation (5.2)]{Demmel-AppliedNumericalLinearAlgebra}.
Using \(\norm{\mat{U}\vek{e}}_2 = \norm{\vek{e}}_2\), this implies
\begin{align*}
  \norm{\mat{U}^T \mat{W}^T \mat{W} \mat{U} - \mat{I}_d}_2
  &= \max_{\vek{e} \in \R^d, \norm{\vek{e}}_2 = 1} \Abs{\vek{e}^T \Paren{\mat{U}^T \mat{W}^T \mat{W} \mat{U} - \mat{I}_d} \vek{e} } \\
  &= \max_{\vek{e} \in \R^d, \norm{\vek{e}}_2 = 1} \Abs{{(\mat{U}\vek{e})}^T \mat{W}^T \mat{W} (\mat{U}\vek{e}) - \vek{e}^T \vek{e}} \\
  &= \max_{\vek{s} \in \mathcal{S}, \norm{\vek{s}}_2 = 1} \Abs{\vek{s}^T \mat{W}^T \mat{W} \vek{s} - 1} \\
  &= \max_{\vek{s} \in \mathcal{S}, \norm{\vek{s}}_2 = 1} \Abs{\norm{\mat{W}\vek{s}}_2^2 - 1}.
\end{align*}
and proves the equivalence of statements 3 and 4.
\end{proof}

Using the QR decomposition of \(\mat{B}\) it is always possible to design a \enquote{perfect} subspace embedding with \(\e = 0\) and \(w = m\)
but computing this QR decomposition is expensive.
Instead, we can use random matrices \(\mat{W}\) which have a high chance of being a subspace embedding.

\begin{definition}\label{def:oblivious-subspace-embedding}
Let \(m, n, w \in \N\) and \(\delta, \e \in (0, 1)\) be given.
A probability distribution \(\Pi\) over matrices \(\mat{W} \in \R^{w \times n}\) is an \emph{oblivious \(\ell_2\)-subspace embedding} (OSE) if, for any matrix \(\mat{B} \in \R^{n \times m}\) a matrix \(\mat{W}\) drawn from \(\Pi\) is a \((1\pm\e)\) \(\ell_2\)-subspace embedding with probability at least \(1 - \delta\).
\end{definition}

Because we will later compute a decomposition of the sketched matrix \(\mat{W}\mat{B}\) we want \(w\) to be as small as possible.
A classic result in sketching is that there are OSEs with \(w = O((m + \log(1/\delta))\e^{-2})\) without any dependence on \(n\).
\Textcite[Theorem 6]{Woodruff-Sketching} shows how this follows from the so-called Johnson-Lindenstrauss Lemma and that it can be achieved by choosing the matrix entries to be properly normalized independent Gaussian random variables.
As a matrix \(\mat{W}\) with Gaussian entries is dense computing the matrix product \(\mat{W}\mat{B}\) takes \(O(w \nnz(\mat{B}))\) operations.
A sparse OSE where all sampled matrices only have \(s\) nonzero entries per column can improve the cost of this product to \(O(s \nnz(\mat{B}))\).
The following sparse OSE result is taken from~\cite{Cohen-NearlyTightObliviousSubspaceEmbeddings}.
\Textcite{Avron-FasterRandomizedInfeasibleIPMs} use the sketching result from~\cite{Cohen-OptimalApproximateMatrixProduct} but we replaced it by a simpler one giving the same complexity guarantees in the end.

\begin{definition}[\cite{CartisFialaShao-EfficientSparseSketchingForLargeScaleLinearLeastSquares}]\label{def:s-hashing-matrix}
An \emph{\(s\)-hashing matrix} \(\mat{W} \in \R^{w \times n}\) is constructed in the following way:
For each column \(s\) entries are sampled randomly without replacement and for each such entry its value is randomly drawn from \(\{ - \frac{1}{\sqrt{s}}, \frac{1}{\sqrt{s}} \}\).
All other entries of \(\mat{W}\) are set to zero.
This defines a probability distribution over \(\R^{w \times n}\) where each sampled matrix has exactly \(s \cdot n\) nonzero entries.
\end{definition}

\begin{theorem}[Theorem 4.2 in~\cite{Cohen-NearlyTightObliviousSubspaceEmbeddings}]\label{thm:sparse-ose}
For any \(\delta < 1/2, \ \e < 1/2\) and \(m, n \in \N\) there exist \(s, w \in \N\) with
\[ w = O \Paren{\frac{m \log(m/\delta)}{\e^2}} \quad \text{and} \quad s = O \Paren{\frac{\log(m/\delta)}{\e}} \]
such that the \(s\)-hashing matrices form an oblivious \(\ell_2\)-subspace embedding.
\end{theorem}

\section{Sketching-Based Preconditioning}\label{sec:sketching-preconditioning}

To simplify the runtime bounds and the following discussion, we will from now on assume that \(\e < 1/2\) is fixed and furthermore assume that \(\mat{W} \in \R^{w \times n}\) is a \(s\)-hashing matrix and a \((1\pm\e)\) \(\ell_2\)-subspace embedding of the column space of \(\mat{B} \in \R^{n \times m}\).
By \cref{thm:sparse-ose} \(w\) and \(s\) can be chosen such that the latter assumption holds with probability at least \(1 - \delta\) and 
\begin{equation} \label{eqn:sparse-ose-params}
  w = O \Paren{m \log(m/\delta)} \quad \text{and} \quad s = O \Paren{\log(m/\delta)}.
\end{equation}

How can these sketching results help with our goal of solving \(\mat{A} \mat{D}^2 \mat{A}^T \Delta\vek{y} = \vek{p}\) with an iterative method?
We will first show how a decomposition of the sketched matrix \(\mat{W}\mat{B}\) can be used to determine a two-sided preconditioner for \(\mat{B}^T \mat{B}\) which reduces the condition number down to a constant and then see how this can be applied to the equation above.

\Textcite{Avron-FasterRandomizedInfeasibleIPMs} use an SVD of the sketched matrix \(\mat{W}\mat{B}\) to get a two-sided preconditioner but we decided to construct it using the computationally cheaper QR decomposition \(\mat{Q}\mat{R} = \mat{W}\mat{B}\).
Here \(\mat{R} \in \R^{m \times m}\) is upper triangular and \(\mat{Q} \in \R^{w \times m}\) has orthonormal columns.
To ensure that the QR decomposition exists we need that \(\mat{B}\) has full column rank. Then, because \(\e < 1\) none of the elements in the column space of \(\mat{B}\) can be in the nullspace of \(\mat{W}\) so \(\rank(\mat{W}\mat{B}) = m\).
The proposed preconditioned matrix has the form
\begin{equation}\label{eqn:preconditioned-matrix}
 \mat{R}^{-T} \mat{B}^T \mat{B} \mat{R}^{-1}.
\end{equation}
Note that applying the preconditioner from the left and from the right in this way ensures that the matrix stays symmetric positive definite and we can apply the CG algorithm to any linear system involving this matrix.
Moreover, because \(\mat{R}\) is upper triangular multiplying \(\mat{R}^{-1}\) or \(\mat{R}^{-T}\) with a vector can be done efficiently in \(O(m^2)\) using a triangular solve.
The following theorem shows that this preconditioner leads to bounds on the condition number of the matrix which can be used to bound the number of CG iterations needed to reach a certain error estimate.

\begin{lemma}\label{thm:condition-number-bound}
Suppose \(\mat{W}\) is a \((1\pm\e)\) \(\ell_2\)-subspace embedding for \(\mat{B}\) and \(\mat{Q}\mat{R}\) is the QR decomposition of \(\mat{W}\mat{B}\).
The singular values of \(\mat{R}^{-T}\mat{B}^T\) and \(\mat{B}\mat{R}^{-1}\) are bounded by
\[ \frac{1}{1 + \e} \leq \sigma_i^2(\mat{R}^{-T}\mat{B}^T) = \sigma_i^2(\mat{B}\mat{R}^{-1}) \leq \frac{1}{1 - \e} \quad \text{for } i = 1, \ldots, m \]
which implies the following bound on the condition number of their product:
\[ \kappa_2(\mat{R}^{-T} \mat{B}^T \mat{B} \mat{R}^{-1}) \leq \frac{1+\e}{1-\e}\]
\end{lemma}
\begin{proof}
The matrices \(\mat{R}^{-T}\mat{B}^T\) and \(\mat{B}\mat{R}^{-1}\) are transposes of each other and hence have the same singular values.
By the Courant-Fisher minmax theorem (e.g.~\cite[Theorem 3.1.2]{HornJohnson-TopicsInMatrixAnalysis})
the minimum and maximum squared singular values of \(\mat{B}\mat{R}^{-1}\) are the same as the minimum and maximum values of \( \norm{\mat{B}\mat{R}^{-1} \tilde{\vek{z}}}_2^2 / \norm{\tilde{\vek{z}}}_2^2\) over all \(\tilde{\vek{z}} \in \R^n\).
By \cref{def:subspace-embedding} applied to \(\vek{z} = \mat{R}^{-1} \tilde{\vek{z}}\) we have
\[
  (1-\e) \norm{\mat{B}\mat{R}^{-1} \tilde{\vek{z}}}_2^2
  \leq \norm{\mat{W}\mat{B}\mat{R}^{-1}\tilde{\vek{z}}}_2^2
  = \norm{\mat{Q}\mat{R} \mat{R}^{-1}\tilde{\vek{z}}}_2^2
  = \norm{\mat{Q}\tilde{\vek{z}}}_2^2
  = \norm{\tilde{\vek{z}}}_2^2
\]
for all \(\tilde{\vek{z}} \in \R^m\) and similarly \((1+\e)\norm{\mat{B}\mat{R}^{-1} \tilde{\vek{z}}}_2^2 \geq \norm{\tilde{\vek{z}}}_2^2\).
Therefore
\[ \frac{1}{1 + \e} \leq \frac{\norm{\mat{B}\mat{R}^{-1} \tilde{\vek{z}}}_2^2}{\norm{\tilde{\vek{z}}}_2^2} \leq \frac{1}{1 - \e} \]
which proves the bounds on the singular values.
The condition number of a matrix is defined as the largest singular value divided by the smallest singular value.
Because \(\mat{R}^{-T} \mat{B}^T \mat{B} \mat{R}^{-1} = {(\mat{B}\mat{R}^{-1})}^T (\mat{B}\mat{R}^{-1})\) the singular values of this matrix are just the squares of the singular values of \(\mat{B}\mat{R}^{-1}\), so
\[ \kappa_2(\mat{R}^{-T} \mat{B}^T \mat{B} \mat{R}^{-1})
   = \frac{\sigma_{\max}^2(\mat{B}\mat{R}^{-1})}{\sigma_{\min}^2(\mat{B}\mat{R}^{-1})} \leq \frac{1+\e}{1-\e} \]
and we are done.
\end{proof}

\begin{remark}
  Note that the condition number bound has \emph{no dependence} on the condition number of \(\mat{B}\) but is bounded by a constant that depends only on \(\e\).
\end{remark}

\begin{lemma}\label{thm:cg-residual-bound}
Suppose \(\mat{W}\) is a \((1\pm\e)\) \(\ell_2\)-subspace embedding for \(\mat{B}\) for \(\e < 1/2\) and \(\mat{Q}\mat{R}\) is the QR decomposition of \(\mat{W}\mat{B}\).
Let \(\vek{q}^j\) be the iterates when solving \[ \mat{R}^{-T} \mat{B}^T \mat{B} \mat{R}^{-1} \vek{q} = \mat{R}^{-T} \vek{p} \] with the CG algorithm and let \(\vek{f}^j\) be the residuals defined by
\( \vek{f}^j = \mat{R}^{-T} \mat{B}^T \mat{B} \mat{R}^{-1} \vek{q}^j - \mat{R}^{-T} \vek{p} \).
Then the \(\ell_2\)-norm of the residuals converges at least linearly to zero:
\[ \norm{\vek{f}^{j+1}}_2 \leq \frac{\e}{1-\e} \norm{ \vek{f}^j }_2 \text{ for all } j \in \N \]
\end{lemma}
\begin{proof}
Theorem 8 of~\cite{Bouyouli-ConjugateGradientConvergence} showed that
\[ \norm{\vek{f}^{j+1}}_2 \leq \frac{\kappa_2(\mat{R}^{-T} \mat{B}^T \mat{B} \mat{R}^{-1}) - 1}{2} \norm{\vek{f}^j}_2 .\]
Using the bound on the condition number from \cref{thm:condition-number-bound} and \(\e < 1/2\) we obtain
\[ \frac{\kappa_2(\mat{R}^{-T} \mat{B}^T \mat{B} \mat{R}^{-1}) - 1}{2}
   \leq \frac{\frac{1 + \e}{1 - \e} - 1}{2} = \frac{\e}{1-\e} < 1 \]
which proves the claim.
\end{proof}

The previous two theorems show that computing the QR decomposition of a sketched matrix \(\mat{W}\mat{B}\) gives an effective way to solve any linear system of the form
\( \mat{B}^T \mat{B}\tilde{\vek{q}} = \vek{p} \)
by replacing it with
\begin{equation}\label{eqn:general-preconditioned-linear-system}
  \mat{R}^{-T}\mat{B}^T\mat{B}\mat{R}^{-1}\vek{q} = \mat{R}^{-T}\vek{p} \quad \text{and} \quad \tilde{\vek{q}} = \mat{R}^{-1}\vek{q}
\end{equation}
and solving the first part using the CG method.
Applying this technique to \cref{eqn:normal-equation} we get
\begin{equation}\label{eqn:preconditioned-normal-equation}
  \mat{R}^{-T}\mat{A}\mat{D}^2\mat{A}^T\mat{R}^{-1}\vek{q} = \mat{R}^{-T}\vek{p} \quad \text{and} \quad \Delta\vek{y} = \mat{R}^{-1}\vek{q}
\end{equation}
by setting \(\mat{B} = \mat{D}\mat{A}^T\).

\section{Choosing the Perturbation Vector}\label{sec:sektching-perturbation-vector}

Now, as mentioned in \cref{chap:ipm} we need to define a perturbation vector \(\vek{v} \in \R^n\) to shift the error term in the Newton system.
Because the CG method is applied to \cref{eqn:preconditioned-normal-equation} an error term \(\vek{f} = \mat{R}^{-T}\mat{A}\mat{D}^2\mat{A}^T\mat{R}^{-1}\vek{q} - \mat{R}^{-T}\vek{p}\) propagates so that
\begin{equation}
  \mat{A} \mat{D}^2 \mat{A}^T \hat{\Delta\vek{y}} = \vek{p} + \mat{R}^T\vek{f}
\end{equation}
and the perturbation vector has to satisfy \(\mat{A}\mat{S}^{-1}\vek{v} = \mat{R}^T \vek{f}\) for a given \(\vek{f}\).
Similar to the method of \textcite{Avron-FasterRandomizedInfeasibleIPMs}, we choose this vector by utilising the previously computed decomposition of the sketched matrix \(\mat{W}\mat{D}\mat{A}^T\).
This also allows us to bound the size of the perturbation vector by the size of \(\vek{f}\).

\begin{lemma}\label{thm:perturbation-vector}
Suppose \(\mat{W}\) is a \(s\)-hashing matrix and \(\mat{Q}\mat{R}\) is the QR decomposition of \(\mat{W}\mat{D}\mat{A}^T\).
Given a vector \(\vek{f} \in \R^m\) the vector
\[ \vek{v} = {(\mat{S}\mat{X})}^{1/2} \mat{W}^T \mat{Q} \vek{f} \]
satisfies \( \mat{A}\mat{S}^{-1}\vek{v} = \mat{R}^T \vek{f} \) and \(\norm{\vek{v}}_2 \leq n \sqrt{\mu} \norm{\vek{f}}_2\).
\end{lemma}
\begin{proof}
Using the definition of \(\mat{Q}\) and \(\mat{R}\) we obtain
\begin{align*}
  \mat{A} \mat{S}^{-1} \vek{v}
  &= \mat{A} \mat{S}^{-1} {(\mat{S}\mat{X})}^{1/2} \mat{W}^T \mat{Q} \vek{f}
   = \mat{A} \mat{D} \mat{W}^T \mat{Q} \vek{f} \\
  &= {(\mat{Q} \mat{R})}^T \mat{Q} \vek{f}
   = \mat{R}^T \mat{Q}^T \mat{Q} \vek{f}
   = \mat{R}^T \vek{f}
\end{align*}
which proves the first part.
To bound the norm consider
\[ 
  \norm{\vek{v}}_2
  = \norm{{(\mat{S}\mat{X})}^{1/2} \mat{W}^T \mat{Q} \vek{f}}_2
  \leq \norm{{(\mat{S}\mat{X})}^{1/2}}_2 \norm{\mat{W}^T}_2 \norm{\mat{Q} \vek{f}}_2.
\]
Now, \(\norm{{(\mat{S}\mat{X})}^{1/2}}_2 \leq \norm{{(\mat{S}\mat{X})}^{1/2}}_F = \sqrt{n \mu}\) by definition of \(\mu\) and \(\norm{\mat{Q}\vek{f}}_2 = \norm{\vek{f}}_2\) because \(\mat{Q}\) has orthonormal columns.
Lastly, we need to bound \(\norm{\mat{W}^T}_2\).
Remember that \(\mat{W}\) is constructed by choosing exactly \(s\) nonzero entries per column and each of these entries is drawn from \(\{-\frac{1}{\sqrt{s}}, \frac{1}{\sqrt{s}}\}\).
Let \(\mathcal{J}_i\) denote the set of indices of the nonzero entries of column \(i\) in \(\mat{W}\) and let \(\vek{z}_{\mathcal{J}_i} \in \R^s\) denote the restriction of \(\vek{z} \in \R^w\) to these indices.
Using standard norm inequalities
\begin{align*}
  \norm{\mat{W}^T \vek{z}}_2^2 &= \sum_{i=1}^{n} \Paren{ \sum_{j=1}^{w} \mat{W}_{ji} \vek{z}_j }^2 = \sum_{i=1}^{n} \Paren{ \sum_{j \in \mathcal{J}_i} \mat{W}_{ji} \vek{z}_j }^2 \\
  &\leq \sum_{i=1}^{n} \Paren{ \sum_{j \in \mathcal{J}_i} \abs{\mat{W}_{ji}} \abs{\vek{z}_j} }^2 = \sum_{i=1}^{n} \Paren{ \frac{1}{\sqrt{s}} \norm{\vek{z}_{\mathcal{J}_i}}_1 }^2 \\
  &\leq \sum_{i=1}^{n} \Paren{ \frac{1}{\sqrt{s}} \sqrt{s} \norm{\vek{z}_{\mathcal{J}_i}}_2 }^2 = \sum_{i=1}^{n} \norm{\vek{z}_{\mathcal{J}_i}}_2^2 \\
  &\leq n \norm{\vek{z}}_2^2
\end{align*}
which implies \(\norm{\mat{W}^T}_2 \leq \sqrt{n}\).
Combining these results,
\[ 
  \norm{\vek{v}}_2
  \leq \norm{{(\mat{S}\mat{X})}^{1/2}}_2 \norm{\mat{W}^T}_2 \norm{\mat{Q} \vek{f}}_2
  \leq \sqrt{n \mu} \sqrt{n} \norm{\vek{f}}_2 = n \sqrt{\mu} \norm{\vek{f}}_2
\]
as claimed.
\end{proof}

Note that compared to the corresponding result in Lemma 4 of~\cite{Avron-FasterRandomizedInfeasibleIPMs} the bound \(\norm{\vek{v}}_2 \leq n \sqrt{\mu} \norm{\vek{f}}_2\) is worse by a factor of \(\sqrt{n}\) but the proof does not introduce any additional randomness and this change does not affect any asymptotic runtime bounds.

\section{Computing an Approximate Newton Direction}\label{sec:sketching-approximate-newton}

The previous considerations give rise to \cref{alg:newton-direction} that tries to determine an approximate Newton direction as in \cref{eqn:perturbed-search-direction} with \(\norm{\vek{v}}_2 \leq \e_{\vek{v}}\).
The behavior of this algorithm in the case that \(\mat{W}\) is not a subspace embedding for \(\mat{D}\mat{A}^T\) will not be analysed, may not be well-defined and will be referred to as a \emph{failure case}.
It is unclear whether there is an efficient way to recognize such a sketching failure but in any case choosing \(\delta\) appropriately later ensures that this will be an exceptionally rare occurence.

\begin{algorithm}
  \SetKwInOut{Input}{Input}
  \Input{
    \(\mat{A} \in \R^{m \times n}\) with rank \(m\),
    \(\vek{b} \in \R^m\), \(\vek{c} \in \R^n\),
    \(\vek{x}, \vek{s} \in \R^n_{>0}\),
    \(\vek{y} \in \R^m\),
    \(\sigma \in (0, 1)\),
    sketching tolerance \(\e \in (0, \frac{1}{2})\),
    failure probability \(\delta \in (0, \frac{1}{2})\),
    error tolerance \(\e_{\vek{v}} > 0\)
  }
  Compute \(\vek{r}_p = \mat{A}\vek{x} - \vek{b}\), \(\vek{r}_d = \mat{A}^T \vek{y} + \vek{s} - \vek{c}\), \(\mat{X} = \diag(\vek{x})\), \(\mat{S} = \diag(\vek{s})\)\;\label{line:compute-residuals}
  Compute \(\vek{p} = -\vek{r}_p + \mat{A}(-\mat{S}^{-1}\mat{X}\vek{r}_d + \vek{x} - \sigma \mu \mat{S}^{-1} \vek{1})\)\;\label{line:compute-p}
  Construct a random \(s\)-hashing matrix \(\mat{W} \in \R^{w \times m}\) as in \cref{thm:sparse-ose} for the given \(\delta\) and \(\e\)\;\label{line:draw-sketching-matrix}
  Compute the QR decomposition \(\mat{Q} \mat{R} = \mat{W} \mat{D} \mat{A}^T\)\;\label{line:compute-qr}
  Run the CG method with \(\vek{q}^0 = \vek{0}\) until \(\mat{R}^{-T} \mat{A} \mat{D}^2 \mat{A}^T \mat{R}^{-1} \vek{q} = \mat{R}^{-T} \vek{p} + \vek{f}\) for some error \(\vek{f}\) satisfying \(\norm{\vek{f}}_2 \leq \e_{\vek{v}}/(n \sqrt{\mu})\)\;\label{line:cg}
  Compute \(\vek{v} = {(\mat{S} \mat{X})}^{1/2} \mat{W}^T \mat{Q} \vek{f}\)\;\label{line:compute-v}
  Compute \(\hat{\Delta\vek{y}} = \mat{R}^{-1}\vek{q}\)\;\label{line:compute-delta-y}
  Compute \(\hat{\Delta\vek{s}} = -\vek{r}_d - \mat{A}^T \hat{\Delta\vek{y}}\)\;\label{line:compute-delta-s}
  Compute \(\hat{\Delta\vek{x}} = -\vek{x} + \sigma \mu \mat{S}^{-1} \vek{1} - \mat{S}^{-1}\mat{X} \hat{\Delta\vek{s}} - \mat{S}^{-1} \vek{v}\)\;\label{line:compute-delta-x}
  \Return{\((\hat{\Delta\vek{x}}, \hat{\Delta\vek{y}}, \hat{\Delta\vek{s}})\)}\;
  \caption{Approximate Newton direction}\label{alg:newton-direction}
\end{algorithm}

\begin{theorem}\label{thm:approximate-newton-convergence}
  Assume that \(\gamma \in (0, 1)\), \(\sigma \in (0, 1)\) and \(\e \in (0, \frac{1}{2})\) are constant, that the initial point \((\vek{x}^0, \vek{y}^0, \vek{s}^0) \in \mathcal{G}\) satisfies \(\vek{x}^0 \geq \vek{x}^*\) and \(\vek{s}^0 \geq \max \{\vek{s}^*, \abs{\mat{A}^T \vek{y}^0 - \vek{c}}\}\) for some \((\vek{x}^*, \vek{y}^*, \vek{s}^*) \in \mathcal{F}^*\) and that \(\mat{A}\) has full row rank.
  Let \((\vek{x}, \vek{y}, \vek{s}) \in \mathcal{N}_{-\infty}(\gamma)\) be such that \(\vek{r} = \eta \vek{r}^0\) for some \(\eta \in [0, 1]\).
  \cref{alg:newton-direction} computes an approximation of the Newton direction that satisfies \cref{eqn:perturbed-search-direction} with \(\norm{\vek{v}}_2 \leq \e_{\vek{v}}\) with probability at least \(1 - \delta\).
  Moreover, if the algorithm does not fail, \cref{line:cg} only needs \(O(\log(n^2 \mu / \e_{\vek{v}}))\) CG iterations and the whole algorithm needs \(O(\log(m/\delta) (\nnz(\mat{A}) + m^3) + \log(n^2 \mu / \e_{\vek{v}})(\nnz(\mat{A}) + m^2))\) operations assuming \(n \leq \nnz(\mat{A})\).
\end{theorem}